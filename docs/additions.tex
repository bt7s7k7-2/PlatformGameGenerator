\usection{Príloha A: Návod k použitiu programu}

\subsection*{Použitie programu priamo}

Program potrebuje k svojej činnosti zdroje. Očakáva že bude mať k dispozíií priečinok \texttt{assets} umiestnený vo working directory. Zdroje sú uložené v globálnych objektoch ktoré treba načítať.

\begin{lstlisting}[language=python]
import pygame
from pg_gen.generation.RoomPrefabRegistry import RoomPrefabRegistry
from pg_gen.level_editor.ActorRegistry import ActorRegistry
from pg_gen.support.constants import ROOM_FOLDER

pygame.init()
ActorRegistry.load_actors()
RoomPrefabRegistry.load(ROOM_FOLDER)
\end{lstlisting}

Ďalším krokom je zvoliť si obtiažnosť ktorú vyžadujete. Parametre majú následovný výzam:

\begin{itemize}
    \item \texttt{JUMP} ⇒ počet skokov v ceste
    \item \texttt{REWARD} ⇒ počet drahokamov v ceste
    \item \texttt{ENEMY} ⇒ počet nepriateľov v ceste (obtiažnejší nepriateľia sa rátajú ako viacej nepriateľov)
    \item \texttt{SPRAWL} ⇒ dĺžka cesty
\end{itemize}

\begin{lstlisting}[language=python]
from pg_gen.generation.RoomParameter import UNUSED_PARAMETER
from pg_gen.difficulty.DifficultyReport import DifficultyReport

target_difficulty = DifficultyReport()
target_difficulty.set_all_parameters(UNUSED_PARAMETER)
target_difficulty.set_parameter(RoomParameter.REWARD, 500)
target_difficulty.set_parameter(RoomParameter.JUMP, 10)
target_difficulty.set_parameter(RoomParameter.ENEMY, 100)
target_difficulty.set_parameter(RoomParameter.SPRAWL, 50)
\end{lstlisting}



Následne je potrebné vytvoriť optimizátor, a vložiť do neho zvolené parametre. Optimizátor ďalej príjima argumenty \texttt{random}, kde je potrebné vložiť náhodný generátor - tu je možné nastaviť konsistetný seed. Optimizátor má možnosť nastaviť populáciu cez parameter \texttt{max\_population} a počet iterácií cez parameter \texttt{max\_generations}.

\begin{lstlisting}[language=python]
from random import Random
from pg_gen.game_core.Universe import Universe
from pg_gen.difficulty.DifficultyOptimizer import DifficultyOptimizer

universe = Universe()
optimizer = DifficultyOptimizer(universe, target_difficulty=target_difficulty, random=Random(108561))
\end{lstlisting}

Nie je potrebné špecifikovať všetky parametre, nepoužité parametre ostanú ako \texttt{UNUSED\_PARAMETER}. Tieto parametre budú mať po optimizácií náhodnú hodnotu, pravdepodobne je múdre ich nastaviť na konštatnú hodnotu. Takto je možné nastaviť všetky parametre, ktoré sa menia pri optimalzácií.

\begin{lstlisting}[language=python]
optimizer.get_parameter(RoomParameter.REWARD).override_value(0.5)
\end{lstlisting}

Po zvolení parametrov a obtiažnosti je možné spustiť optimizátor.

\begin{lstlisting}[language=python]
optimizer.initialize_population()
optimizer.optimize()
\end{lstlisting}

Po optimizácií je možné vybrať najbližšieho kandidáta a použiť jeho úroveň. Následne je potrebné aktivovať koreňovú miestnosť a vložiť do nej objekt hráča.

\begin{lstlisting}[language=python]
from pg_gen.generation.RoomController import RoomController
from pg_gen.actors.Player import Player
from pg_gen.support.Point import Point
from pg_gen.support.constants import ROOM_HEIGHT, ROOM_WIDTH

best_candidate = optimizer.get_best_candidate()
map = best_candidate.get_map()
universe.map = map

room_controller = RoomController.initialize_and_activate(universe, map.get_room(Point.ZERO), None)
room_controller.world.add_actor(Player(position=Point(ROOM_WIDTH / 2, ROOM_HEIGHT / 2)))
\end{lstlisting}

Pre použitie hry interaktívne stači spustiť \texttt{InteractiveGameLoop}.

\begin{lstlisting}[language=python]
from pg_gen.game_core.InteractiveGameLoop import InteractiveGameLoop

game_loop = InteractiveGameLoop(universe)
game_loop.run()
\end{lstlisting}

Pre použitie s modelom je potrebné vytvoriť vlastný pygame-ový \texttt{Surface} a vložiť ho do \texttt{GameLoop}. Následne je možné vkladať vstup cez objekt \texttt{InputState} a manuálne simulovať update. Keďže mimo interaktívneho prístupu nie je herná sluťka naviazaná na reálny čas, je potrebné stanoviť čas, ktorý prejde medzi každý update krokom. Napríklad: ak chceme mať 20 krokov cez jednu simulovanú sekundu, tak čas medzi krokmi je $ 1 / 20 $ sekúnd.

\begin{lstlisting}[language=python]
surface = pygame.display.set_mode((CAMERA_SCALE * ROOM_WIDTH, CAMERA_SCALE * ROOM_HEIGHT))
self.game_loop = GameLoop(surface, self.universe)

input_state = self.universe.di.inject(InputState)
input_state.clear()

input_state.left = True
input_state.jump = True
# [...]

game_loop.update_and_render(1 / fps)
\end{lstlisting}

\subsection*{Použitie Gymnasium prostredia}

Gymnasium prostredie rieši manažment zdrojov automaticky. Pre jeho fungovanie je potrebné iba vložiť chcené parametre. 
